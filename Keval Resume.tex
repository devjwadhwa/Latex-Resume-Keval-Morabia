\documentclass[]{Keval-resume}
\usepackage{fancyhdr}
\usepackage{hyperref}

\pagestyle{fancy}
\fancyhf{}
\renewcommand{\headrulewidth}{0pt}

\hypersetup{%
 	colorlinks=false,% hyperlinks will be black
 	linkbordercolor=red,% hyperlink borders will be red
 	pdfborderstyle={/S/U/W 0.25}% border style will be underline of width 0.25pt
 }
 
\begin{document}

%%%%%%%%%%%%%%%%%%%%%%%%%%%%%%%%%%%%%%
%
%     TITLE NAME
%
%%%%%%%%%%%%%%%%%%%%%%%%%%%%%%%%%%%%%%
\namesection{Keval Morabia}{
	\normalfont 217-819-xxxx \textbullet{} Champaign, Illinois 61820 \textbullet{} Email: \href{mailto:morabia2@illinois.edu}{morabia2@illinois.edu} \\
	
	\normalfont \href{https://github.com/kevalmorabia97}{\textbf{github}.com/kevalmorabia97} \textbullet{} \href{https://linkedin.com/in/kevalmorabia97}{ \textbf{linkedin}.com/in/kevalmorabia97}
}

\descript{}
\postsectionsep

%%%%%%%%%%%%%%%%%%%%%%%%%%%%%%%%%%%%%%
%     EDUCATION
%%%%%%%%%%%%%%%%%%%%%%%%%%%%%%%%%%%%%%
\section{Education}
\hrulefill
\postsectionsep

\subsection{University of Illinois at Urbana-Champaign (UIUC) \hfill \normalfont U\lowercase{rbana}, IL}
\position{Master of Computer Science \normalfont (GPA: N.A.)}{Aug 2019 - Dec 2020}
\textbf{Coursework:} Natural Language Processing, Advanced Information Retrieval 
\sectionsep

\subsection{Birla Institute of Technology and Science (BITS), Pilani \hfill \normalfont H\lowercase{yderabad}, I\lowercase{ndia}}
\position{B.E. (Hons.) in Computer Science \normalfont (CGPA: 9.72/10)}{Aug 2015 - May 2019}
\textbf{Electives:} Artificial Intelligence, Machine Learning, Information Retrieval, Data Mining, Number Theory, Cryptography
%\textbf{Teaching Exp:} Data Mining \textbullet{} Object-oriented Programming

\sectionsep

%%%%%%%%%%%%%%%%%%%%%%%%%%%%%%%%%%%%%%
%     RESEARCH EXPERIENCE
%%%%%%%%%%%%%%%%%%%%%%%%%%%%%%%%%%%%%%
\section{Research Experience} 
\hrulefill
\postsectionsep 

\subsection{UIUC - Forward Data Lab \hfill \normalfont U\lowercase{rbana}, IL}
\position{Research Assistant}{Aug 2019 - Present}
\textbullet{} Working on Web Information Extraction with \href{http://www.forwarddatalab.org/kevinccchang}{\textbf{Prof. Kevin Chang}}; funded by NSF and IBM \\
\textbullet{} Implemented a novel model in \textbf{PyTorch} to extract product details from a web page using \textbf{Computer Vision} based Web Object Extraction \\
\textbullet{} Created a \textbf{labeled dataset} for extracting price, image and title information of products from a web page
\sectionsep

\subsection{Microsoft Research \hfill \normalfont B\lowercase{engaluru}, I\lowercase{ndia}}
\position{Research Intern}{Jan 2019 - Jul 2019}
\textbullet{} Implemented \textbf{Graph Recurrent Neural Networks} from scratch in \textbf{TensorFlow} to \textbf{Learn Embeddings} for \textbf{300,000} entities in a heterogeneous graph \\
\textbullet{} Collaborated with a \textbf{team of 10} members to design a novel Deep Neural Net architecture \\
\textbullet{} Utilized embeddings for \textbf{recommending important messages} to a user posted in a channel on MSTeams
%\textbullet{} Achieved comparable or better results than \textbf{Graph Convolutional Networks} on \textbf{8} benchmark datasets for node classification, ranking and rating prediction tasks
\sectionsep

\subsection{BITS Pilani \hfill \normalfont H\lowercase{yderabad}, I\lowercase{ndia}}
\position{Research Assistant}{Jan 2018 - Dec 2018} 
\textbullet{} Analyzed \textbf{Twitter Stream} for \textbf{Event Detection} by leveraging Wikipedia \\
\textbullet{} Segmented tweets and hash-tags; applied Jarvis-Patrick clustering; summarized newsworthy events in \textbf{Python}\\
\textbullet{} Achieved a precision of \textbf{88.12\%} (Improvement of \textbf{8\%})
\sectionsep

%\subsection{BITS P\lowercase{ilani} \hfill \normalfont H\lowercase{yderabad}, I\lowercase{ndia}}
%\position{Study Project}{Aug 2018 - Nov 2018}
%\textbullet{} Made a unified interface for \textbf{Querying Heterogeneous Databases} consisting multiple relational \textbf{(MySQL)} and object-relational \textbf{(ObjectDB)} databases at different physical locations \\
%\textbullet{} Eliminated the work of querying multiples databases individually and manually combining the results
%\sectionsep

%%%%%%%%%%%%%%%%%%%%%%%%%%%%%%%%%%%%%%
%     PROFESSIONAL EXPERIENCE
%%%%%%%%%%%%%%%%%%%%%%%%%%%%%%%%%%%%%% 
\section{Professional Experience} 
\hrulefill
\postsectionsep

\subsection{Arcesium India Private Limited \hfill \normalfont H\lowercase{yderabad}, I\lowercase{ndia}}
\position{Software Engineering Intern}{May 2018 - Jul 2018}
\textbullet{} Enhanced \textbf{Expense Management System} by adding back-end services in \textbf{Java} to modify in \textbf{MS SQL Server} database \\
\textbullet{} Automated comparison of budgeted inputs with actual expenses on a UI implemented in \textbf{JavaScript} \\
\textbullet{} Wrote about \textbf{100} unit test cases in \textbf{JUnit} and \textbf{Mockito} and increased test coverage by \textbf{5\%} \\
\textbullet{} Interacted with clients to cater to specific feature requirements

\sectionsep

%\subsection{WATConsult \hfill \normalfont M\lowercase{umbai}, I\lowercase{ndia}}
%\position{Web Development Intern}{May 2017 - Jul 2017}
%\textbullet{} Designed front-end of a location tracking device using \textbf{HTML}, \textbf{CSS}, and Google Maps \textbf{JavaScript} API
%\sectionsep

%%%%%%%%%%%%%%%%%%%%%%%%%%%%%%%%%%%%%%
%     PUBLICATIONS
%%%%%%%%%%%%%%%%%%%%%%%%%%%%%%%%%%%%%%
\section{Publications} 
\hrulefill
\postsectionsep 

\textbullet{} \textbf{Morabia, K.}, Bhanu Murthy, N. L., Malapati, A., \& Samant, S. (2019, June). \textbf{SEDTWik: Segmentation-based Event Detection from Tweets using Wikipedia}. In Proceedings of the 2019 Conference of the North American Chapter of the Association for Computational Linguistics \textbf{(NAACL)}: Student Research Workshop (pp. 77-85) \href{https://www.aclweb.org/anthology/papers/N/N19/N19-3011/}{[\textbf{Paper}]} \href{https://github.com/kevalmorabia97/SEDTWik-Event-Detection-from-Tweets}{[\textbf{Code}]}
\sectionsep

%%%%%%%%%%%%%%%%%%%%%%%%%%%%%%%%%%%%%%
%     PROJECTS
%%%%%%%%%%%%%%%%%%%%%%%%%%%%%%%%%%%%%%
\section{Independent Projects}
\hrulefill
\postsectionsep

\textbullet{} Created \textbf{pyTweetCleaner}, a \textbf{Python} module for cleaning and parsing of \textbf{Twitter} JSON data

\textbullet{} Built an application in \textbf{JavaFX} for \textbf{Frequent Patterns \& Association Rule Mining} using Apriori algorithm

%\textbullet{} \textbf{Hierarchical Clustering} to construct a phylogenetic tree of evolution from DNA sequences

%\textbullet{} \textbf{ID3 Decision Tree Classifier} along with Reduced Error Pruning and Random Forest generation

%\textbullet{} \textbf{Anomaly Detection} using \textbf{Local Outlier Factor (LOF)} algorithm

%\textbullet{} Rank web pages in a corpus using \textbf{PageRank} algorithm and handling dead ends \& spider traps

%\textbullet{} Developed a Blockchain-based \textbf{Chat Messenger} in \textbf{Java} in a \textbf{team of 5}

\sectionsep

%%%%%%%%%%%%%%%%%%%%%%%%%%%%%%%%%%%%%%
%     SKILLS
%%%%%%%%%%%%%%%%%%%%%%%%%%%%%%%%%%%%%%
\section{Skills} 
\hrulefill
\postsectionsep

\textbullet{} \textbf{Programming:} Java, Python, C, SQL, HTML, CSS, JavaScript, LaTeX

\textbullet{} \textbf{Technical:} TensorFlow, GIT, PyTorch, Java Swing, JavaFX, JUnit, Mockito, Spring Framework, My Batis

%\textbullet{} \textbf{Language:} Gujarati, Hindi (Native)
\sectionsep

%%%%%%%%%%%%%%%%%%%%%%%%%%%%%%%%%%%%%%
%     Leadership and Honors
%%%%%%%%%%%%%%%%%%%%%%%%%%%%%%%%%%%%%%
\section{Leadership and Honors} 
\hrulefill
\postsectionsep 

\begin{minipage}[t]{.78\textwidth}
	\textbullet{} Awarded \textbf{Merit Scholarship} (\textbf{Top 1\%} out of \textbf{700} undergrad students)
\end{minipage}%
\begin{minipage}[t]{.22\textwidth}
	\hfill Aug 2015 - May 2019
\end{minipage}

\begin{minipage}[t]{.78\textwidth}
	\textbullet{} Organized cultural dance events as \textbf{President} of \textbf{150}-membered Gujarati Association - BITS Pilani, Hyderabad; attended by over \textbf{1000} students; raised \textbf{INR 30,000} fund
\end{minipage}%
\begin{minipage}[t]{.22\textwidth}
	\hfill Jul 2017 - May 2018
\end{minipage}

%\sectionsep

%%%%%%%%%%%%%%%%%%%%%%%%%%%%%%%%%%%%%%	
\end{document}  \documentclass[]{article}
